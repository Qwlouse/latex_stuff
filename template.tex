%%%%%%%%%%%%%%%%%%%%%%%%%%%%%%%%% Basic Packages %%%%%%%%%%%%%%%%%%%%%%%%%%%%%%%

\usepackage[utf8]{inputenc} % use utf8 as input encoding
\usepackage[T1]{fontenc}    % use 8-bit T1 fonts
\usepackage[english]{babel} % improved hyphening and internationalization
\usepackage{microtype}      % microtypography!!!


%%% For figures and tables
\usepackage{graphicx}
\usepackage{booktabs}       % professional-quality tables
\usepackage{subcaption}     % new version of subfigures package
\usepackage{rotating}       % sidewaysfigure
\usepackage{wrapfig}        % figures wrapped in text
\usepackage{algorithm2e}    % algorithms [linesnumbered]


%%% Math stuff
\usepackage{amsfonts}       % blackboard math symbols
\usepackage{amsmath}        % equations
\usepackage{nicefrac}       % compact symbols for 1/2, etc.
\usepackage{bm}             % boldmath using \bm also for greek symbols


%%% for custom macros
\usepackage{xspace}         % a sort of conditonal space
\usepackage{ifthen}         % provides conditonals with \ifthenelse{}{}{}
\usepackage[table]{xcolor}	% for defining colors


%%%%%%%%%%%%%%%%%%%%%%%%%%% Citations and References %%%%%%%%%%%%%%%%%%%%%%%%%%%
\usepackage{bibentry}       % place bibliographic entries anywhere in the text
\usepackage{url}            % simple URL typesetting

\usepackage{varioref}    % \vref which can output "Figure 2 on the next page"
\usepackage{hyperref}    % The order is important varioref > hyperref > cleveref > glossaries

% Make the links dark blue like in ICML style
\definecolor{mydarkblue}{rgb}{0,0.08,0.45}
\hypersetup{ %
    breaklinks=true,
    unicode=true,
    pdftitle={Binding Survey},
    pdfauthor={Klaus Greff},
    pdfsubject={},
    pdfkeywords={Binding},
    pdfborder=0 0 0,
    pdfpagemode=UseNone,
    colorlinks=true,
    linkcolor=mydarkblue,
    citecolor=mydarkblue,
    filecolor=mydarkblue,
    urlcolor=mydarkblue,
    pdfview=FitH}
\usepackage{cleveref}    % provide the versatile \cref
\usepackage[all]{hypcap} % solves the problem that refs anchor to caption instead of the float beginning.

% Capitalize section in \autoref
\addto\extrasenglish{
  \def\sectionautorefname{Section}%
  \def\subsectionautorefname{Section}%
  \def\subsubsectionautorefname{Section}%
}

% Capitalize \cref names
\crefname{chapter}{Chapter}{Chapters}
\crefname{section}{Section}{Sections}
\crefname{subsection}{Section}{Sections}
\crefname{subsubsection}{Section}{Sections}
\crefname{figure}{Figure}{Figures}
\crefname{table}{Table}{Tables}
\crefname{subfigure}{Figure}{Figures}
\crefname{equation}{Equation}{Equations}

% Instructions:
% usually you want to use \cref{LABEL}
% \labelcref{} for getting just (number)  <== for example for equations
% \cpageref{} for getting the page

%%%%%%%%%%%%%%%%%%%%%%%%%%%%%%%%% Acronyms %%%%%%%%%%%%%%%%%%%%%%%%%%%%%%%%%%%%%
%%% For Acronyms, Notation and Glossary
% Contrary to common advice that hyperref should be loaded last, glossaries needs to be after hyperref
\usepackage[hyperfirst=false,acronym]{glossaries}

% Command to switch behavior depending on whether inside a float or not
\makeatletter
\newcommand\InFloat[2]{\ifnum\@floatpenalty<0\relax{#1}\else{#2}\fi}%
\makeatother

% helper command to create a glossary entry and corresponding shorthand commands
% {name}{abbrev}{long}{optional citation/explanation}
% For name=RNN this creates commands \RNN, \RNNs, \RNNlong, \RNNfirst
\newcommand{\createAcronymCmdINTERNAL}[4]{%
\newacronym[user1=#4]{#1}{#2}{#3}
\expandafter\newcommand\csname #1\endcsname{\InFloat{\glsfirst{#1}}{\gls{#1}}\xspace}
\expandafter\newcommand\csname #1s\endcsname{\InFloat{\glsfirstplural{#1}}{\glspl{#1}}\xspace}
\expandafter\newcommand\csname #1long\endcsname{\glsdesc{#1}\xspace}
\expandafter\newcommand\csname #1first\endcsname{\glslocalreset{#1}\gls{#1}\xspace}
}

% user-command to create acronyms.
% \createAcronymCmd[name]{abbrev}{long}{citation/explanation}
% works the same as \createAcronymCmdINTERNAL
% but name is optional and defaults to abbrev
\newcommand{\createAcronymCmd}[4][]{%
\ifthenelse { \equal {#1} {} }  %if name not specified use abbrev as name
    { \createAcronymCmdINTERNAL{#2}{#2}{#3}{#4} }   % if #1 == blank
    { \createAcronymCmdINTERNAL{#1}{#2}{#3}{#4} }   % else (not blank)
}


%% A custom style that uses the user1 field for optional description/citation
%% first use:  \textit{LONG}~(SHORT; user1)
%%             \textit{LONG}~(SHORT)
%% other uses: SHORT
\newacronymstyle{optional-cite}% new style name
{%
\glsgenacfmt
}%
{%
\renewcommand*{\firstacronymfont}[1]{\textit{##1}}
% No case change, singular first use:
\renewcommand*{\genacrfullformat}[2]{%
\firstacronymfont{\glsentrylong{##1}}##2%
~(\glsentryshort{##1}\ifglshasfield{user1}{##1}{;~\glsuseri{##1}}{})%
}%
% First letter upper case, singular first use:
\renewcommand*{\Genacrfullformat}[2]{%
\firstacronymfont{\Glsentrylong{##1}}##2%
~(\Glsentryshort{##1}\ifglshasfield{user1}{##1}{;~\glsuseri{##1}}{})%
}%
% No case change, plural first use:
\renewcommand*{\genplacrfullformat}[2]{%
\firstacronymfont{\glsentrylongpl{##1}}##2%
~(\glsentryshortpl{##1}\ifglshasfield{user1}{##1}{;~\glsuseri{##1}}{})%
}%
% First letter upper case, plural first use:
\renewcommand*{\Genplacrfullformat}[2]{%
\firstacronymfont{\Glsentrylongpl{##1}}##2%
~(\Glsentryshortpl{##1}\ifglshasfield{user1}{##1}{;~\glsuseri{##1}}{})%
}%
}

% Enable the custom acronym style
\setacronymstyle{optional-cite}
\glsdisablehyper  % disable hyperlinks to first occurence in acronyms


%%%%%%%%%%%%%%%%%%%%%%%%%%%%% Unicode Characters %%%%%%%%%%%%%%%%%%%%%%%%%%%%%%%
\usepackage{textgreek}      % greek symbols in text (i.e. \textalpha}
\usepackage[verbose]{newunicodechar} % for defining new unicode characters

\newcommand{\textOrMath}[2]{{\ifmmode{#1}\else{#2}\fi}}

% Sets
\newunicodechar{ℕ}{{\ensuremath{\mathbb{N}}}}
\newunicodechar{ℤ}{{\ensuremath{\mathbb{Z}}}}
\newunicodechar{ℝ}{{\ensuremath{\mathbb{R}}}}
\newunicodechar{ℂ}{{\ensuremath{\mathbb{C}}}}

\newunicodechar{⊂}{\subset}
\newunicodechar{∈}{\in}
\newunicodechar{∪}{\cup}
\newunicodechar{∩}{\cap}

% logic
\newunicodechar{∀}{\forall}
\newunicodechar{∃}{\exists}
\newunicodechar{¬}{\neg}
\newunicodechar{∨}{\wedge}
\newunicodechar{∧}{\vee}

% Greek letters
\newunicodechar{α}{{\textOrMath{\alpha}{\textalpha}}}
\newunicodechar{β}{{\textOrMath{\beta}{\textbeta}}}
\newunicodechar{γ}{{\textOrMath{\gamma}{\textgamma}}}
\newunicodechar{δ}{{\textOrMath{\delta}{\textdelta}}}
\newunicodechar{ε}{{\textOrMath{\epsilon}{\textepsilon}}}
\newunicodechar{ζ}{{\textOrMath{\zeta}{\textzeta}}}
\newunicodechar{η}{{\textOrMath{\eta}{\texteta}}}
\newunicodechar{θ}{{\textOrMath{\theta}{\texttheta}}}
\newunicodechar{ι}{{\textOrMath{\iota}{\textiota}}}
\newunicodechar{κ}{{\textOrMath{\kappa}{\textkappa}}}
\newunicodechar{λ}{{\textOrMath{\lambda}{\textlambda}}}
\newunicodechar{μ}{{\textOrMath{\mu}{\textmugreek}}}
\newunicodechar{ν}{{\textOrMath{\nu}{\textnu}}}
\newunicodechar{ο}{{o}}
\newunicodechar{π}{{\textOrMath{\pi}{\textpi}}}
\newunicodechar{ρ}{{\textOrMath{\rho}{\textrho}}}
\newunicodechar{σ}{{\textOrMath{\sigma}{\textsigma}}}
\newunicodechar{τ}{{\textOrMath{\tau}{\texttau}}}
\newunicodechar{υ}{{\textOrMath{\upsilon}{\textupsilon}}}
\newunicodechar{φ}{{\textOrMath{\varphi}{\textvarphi}}}
\newunicodechar{ϕ}{{\textOrMath{\phi}{\textphi}}}
\newunicodechar{χ}{{\textOrMath{\chi}{\textchi}}}
\newunicodechar{ψ}{{\textOrMath{\psi}{\textpsi}}}
\newunicodechar{ω}{{\textOrMath{\omega}{\textomega}}}
\newunicodechar{ξ}{{\textOrMath{\xi}{\textxi}}}
\newunicodechar{ς}{{\textOrMath{\varsigma}{\textvarsigma}}}
% Upper
\newunicodechar{Ω}{{\textOrMath{\Omega}{\textOmega}}}
\newunicodechar{Ξ}{{\textOrMath{\Xi}{\textXi}}}
\newunicodechar{Π}{{\textOrMath{\Pi}{\textPi}}}
\newunicodechar{Ψ}{{\textOrMath{\Psi}{\textPsi}}}
\newunicodechar{Γ}{{\textOrMath{\Gamma}{\textGamma}}}
\newunicodechar{Φ}{{\textOrMath{\Phi}{\textPhi}}}
\newunicodechar{Σ}{{\textOrMath{\Sigma}{\textSigma}}}
\newunicodechar{Δ}{{\textOrMath{\Delta}{\textDelta}}}
\newunicodechar{Θ}{{\textOrMath{\Theta}{\textTheta}}}
\newunicodechar{Λ}{{\textOrMath{\Lambda}{\textLambda}}}

% Arrows
\newunicodechar{⇐}{\Leftarrow}
\newunicodechar{⇔}{\Leftrightarrow}
\newunicodechar{⇒}{\Rightarrow}
\newunicodechar{→}{\rightarrow}
\newunicodechar{↦}{\mapsto}

% Other
\newunicodechar{√}{\srqt}
\newunicodechar{ℵ}{{\aleph}}
\newunicodechar{×}{\times}
\newunicodechar{∂}{\partial}
\newunicodechar{∘}{\circ}
\newunicodechar{∇}{{\nabla}}
\newunicodechar{⊥}{\perp}
\newunicodechar{∡}{\measuredangle}
\newunicodechar{∥}{\|}
\newunicodechar{∞}{{\infty}}
\newunicodechar{∝}{\propto}
\newunicodechar{∅}{{\varnothing}}
\newunicodechar{›}{\rangle}
\newunicodechar{‹}{\langle}


%%%%%%%%%%%%%%%%%%%%%%%%%%%%%%%%% Math Commands %%%%%%%%%%%%%%%%%%%%%%%%%%%%%%%%%%%

% Command to make a BIG fraction in inline-math
\newcommand\ddfrac[2]{\frac{\displaystyle #1}{\displaystyle #2}}

% argmax operator
\DeclareMathOperator*{\argmax}{\arg\!\max}

\DeclareMathOperator*{\sigmoid}{sigmoid}
\DeclareMathOperator*{\softplus}{softplus}

% "defined as" operator (= with a small "def" ontop)
\newcommand{\deq}{\stackrel{\text{\tiny def}}{=}}

% a frac command with partials ∂1/∂2
\newcommand{\deriv}[2]{\frac{\partial #1}{\partial #2}}

\DeclareMathOperator*{\E}{\mathbb{E}}
\DeclareMathOperator*{\Var}{Var}
\DeclareMathOperator*{\Cov}{Cov}

%%%%%%%%%%%%%%%%%%%% Combined List of Figures and Tables %%%%%%%%%%%%%%%%%%%%%%%
% thanks to https://tex.stackexchange.com/a/258508/72180
% can use \listoffigtab to create a combined list of figures and tables
\usepackage[titles]{tocloft}
\newlistof{figtab}{loft}{List of Figures and Tables}
\makeatletter
% Change the file extension of both lot and lof
\def\ext@figure{loft}
\def\ext@table{loft}
% Store the original `\thefigure` and `\thetable`
\let\tohe@thefigure\thefigure
\let\tohe@thetable\thetable
% Redefine them to contain a "dummy" `\tohe@list...`
\def\thefigure{\tohe@listfig\tohe@thefigure}
\def\thetable{\tohe@listtab\tohe@thetable}
% Make the two dummy commands truly dummy
\let\tohe@listfig\relax
\let\tohe@listtab\relax
% Store the original `\listoffigtab`
\let\tohe@listoffigtab\listoffigtab
% Redefine it in such a way that the dummy commands insert "Fig." or "Tab." respectively
\def\listoffigtab{%
  \begingroup
  \def\tohe@listfig{Fig.~}
  \def\tohe@listtab{Tab.~}
  \tohe@listoffigtab
  \endgroup
}
% Change \listoffigtab spacing
\setlength{\cftfignumwidth}{5em}
\setlength{\cfttabnumwidth}{5em}
\setlength{\cftfigindent}{0pt}
\setlength{\cfttabindent}{0pt}
\makeatother


%%%%%%%%%%%%%%%%%%%%%%  Draft stuff  %%%%%%%%%%%%%%%%%%%%%%%%%%%%%%%
\usepackage{lipsum}  % for fake text

%%% for fixme notes
\usepackage{fixme}
\fxsetup{
status=draft,
author=,
layout=pdfmargin, % inline, footnote, or pdfnote
theme=color
}

% fix the marginparwidt
%\usepackage[margin=1.3in, marginparwidth=1in, marginparsep=0.2in]{geometry}

% use the right side for margin notes
%\reversemarginpar

% make font smaller in margin notes
%\fxsetup{marginface=\linespread{0.9}\scriptsize}

% register new author K -> \knote{}
\FXRegisterAuthor{k}{ak}{K}

% Linenumbers
\usepackage[mathlines]{lineno}
\usepackage{linenofix} % fix for usage with amsmath environments (e.g., align)
\renewcommand{\linenumberfont}{\color{gray}\normalfont\footnotesize\sffamily}
% needs a \linenumbers inside the document

%%%%%%%%%%%%%%%%%%%%%%  Other Commands and Settings %%%%%%%%%%%%%%%%%%%%%%%%%%%%%%%
% Command to remove some whitespace from itemize
\newcommand{\noskip}{%
\setlength\itemsep{3pt}%
\setlength\parsep{0pt}%
\setlength\partopsep{0pt}%
\setlength\parskip{0pt}%
\setlength\topskip{0pt}%
}

%% Enable starting chapters on the left side
%\makeatletter
%\@openrightfalse
%\makeatother
